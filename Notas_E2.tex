%----------------------------------------------------------------------------------------
%	PACKAGES AND OTHER DOCUMENT CONFIGURATIONS
%----------------------------------------------------------------------------------------

\documentclass[a4paper]{article}

\usepackage{mhchem}

\usepackage{gensymb}

%\usepackage[super]{natbib}

\usepackage[T1]{fontenc} % Use 8-bit encoding that has 256 glyphs

\usepackage{lmodern}

\usepackage[hyphenbreaks]{breakurl}

\usepackage[hyphens]{url}

%\usepackage[super,sort&compress]{natbib}
%\usepackage{natbib}
%\setlength{\bibsep}{0.0pt}

\usepackage{subcaption}

\usepackage{graphicx}

\linespread{1.05} % Line spacing - Palatino needs more space between lines
\usepackage{microtype} % Slightly tweak font spacing for aesthetics

\usepackage[spanish]{babel} % Language hyphenation and typographical rules

\usepackage[numbib,notlof,notlot,nottoc]{tocbibind} % Shows bibliography as a section

\usepackage[hmarginratio=1:1,top=32mm,columnsep=20pt]{geometry} % Document margins

\usepackage[section]{placeins}

\usepackage{float}

\usepackage{booktabs} % Horizontal rules in tables

\usepackage{enumitem} % Customized lists


\usepackage{abstract} % Allows abstract customization

\renewcommand{\abstractnamefont}{\normalfont\bfseries} % Set the "Abstract" text to bold

\usepackage{fancyhdr} % Headers and footers
\pagestyle{fancy} % All pages have headers and footers
\fancyhead{} % Blank out the default header
\fancyfoot{} % Blank out the default footer
\fancyhead[C]{Apuntes y ejercicios resueltos de Estructura de la Materia 2 $\bullet$ Ignacio Poggi} % Custom header text
\fancyfoot[C]{\thepage} % Custom footer text

\usepackage{titling} % Customizing the title section

\usepackage{hyperref} % For hyperlinks in the PDF

%----------------------------------------------------------------------------------------
%	TITLE SECTION
%----------------------------------------------------------------------------------------

\setlength{\droptitle}{-4\baselineskip} % Move the title up

\pretitle{\begin{center}\LARGE\bfseries} % Article title formatting
\posttitle{\end{center}} % Article title closing formatting
\title{Estructura de la Materia 2} % Article title
\author{%
\textsc{Ignacio Poggi} \\[1ex] % Your name
\normalsize \href{mailto:ignaciop.3@gmail.com}{ignaciop.3@gmail.com} % Your email address
}

\date{\today} % Leave empty to omit a date
\renewcommand{\maketitlehookd}{%
\begin{abstract}
\noindent Apuntes y ejercicios resueltos de Estructura de la Materia 2 (2º cuatrimestre 2022).
\end{abstract}
}

%----------------------------------------------------------------------------------------

\begin{document}
\maketitle

% Print the title

%----------------------------------------------------------------------------------------
%	ARTICLE CONTENTS
%----------------------------------------------------------------------------------------

\section{Gu\'ia 1: Redes Cristalinas y Espacio Rec\'iproco}

\subsection{}

\begin{itemize}
\item En este \'item nos piden describir una estructura c\'ubica centrada en la base (SC con puntos adicionales en las caras horizontales de la celda).


\end{itemize}

Integrando (\ref{eq:primerprincipio}) tenemos

\begin{equation}
\label{eq:primerprincipio2}
\Delta E = W_{ext} + Q = -W + Q
\end{equation}

donde $W$ es el trabajo del sistema y $Q$ es el calor absorbido o cedido por el sistema.


Definimos \textit{capacidad t\'ermica o calor\'ifica} de una sustancia a la cantidad

$$C_{X} = \frac{dQ}{dT}\bigg|_{X}$$

Esto nos da la relaci\'on entre la cantidad infinitesimal de calor absorbido por una sustancia en un proceso reversible $X$ y el incremento infinitesimal de temperatura que sufre. Las cantidades m\'as comunes son $C_{V}$ (a $V = cte$) y $C_{P}$ (a $P = cte$):\\





%----------------------------------------------------------------------------------------
%	REFERENCE LIST
%----------------------------------------------------------------------------------------
\newpage
\begin{thebibliography}{99} % Bibliography - this is intentionally simple in this template
 
\end{thebibliography}


%----------------------------------------------------------------------------------------

\end{document}